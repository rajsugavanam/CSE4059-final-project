\documentclass[12pt]{article}


\usepackage{custom_package}
\usepackage[mocha,styleAll]{catppuccinpalette}
\usepackage[paper=letterpaper, vmargin=1in, hmargin=0.5in]{geometry}


% \fancyhead[L]{(Names)}
\fancyhead[C]{CSE 4059 (GPU Programming)}
\rhead{Raj Sugavanam \& Junseo Shin}
\lhead{CRT}
% \fancyhead[R]{(IDs)}
\setlength{\headheight}{14.5pt}
\pagestyle{fancy}


\fboxsep=1mm
\fboxrule=2pt


\graphicspath{ {.} }


\begin{document}


\begin{center}
    \section*{Project Plan}
\end{center}


\subsection*{Project Abstract}


% PROBLEM DESCRIPTION
We will attempt to implement physically based rendering via ray tracing to
generate an accurate render of an object. For the sake of time, we will focus
on only a few properties of light and one key object in a scene (none or very
few background elements). We will not be baking normals or supporting material
properties that are typically seen when creating materials in a software such
as substance designer or Blender. However we may make triangulated 3d meshes
that are exported from Blender.

\subsection*{Application and Application Domain}
% Application and Application Domain
The application from which this problem originates is ray tracing in video
games. This belongs to graphics programming applications and we believe is
similar in terms of high level principles to collision detection. This is
significant for simulations or realistic modeling.


% ALGORITHMS
\subsection*{Algorithms}
We plan to use a collision detection algorithm to determine whether and where a
light ray hits an object. We either plan to use ray hit detection with a
triangle’s plane (if we use triangulated meshes) or a more hard-coded
mathematical model based on the object geometry we are dealing with if we do
not use triangulated meshes. Light rays will likely start from the camera and
use their reflection to compute how intense the light is at that point.

\subsection*{Implementation Strategy}
% IMPLEMENTATION STRATEGY
We predict that we may have to parse and read an .obj file and store its
triangles in memory. We are anticipating that we will have to code some type of
gaussian elimination due to needing to solve systems of equations. This can be
easily parallelized. An alternative might be to use determinants via co-factor
expansion for only small matrices (due to their n! computation time, we cannot
use large matrices). We also will be heavily dealing with dot products of
3-dimensional vectors, and while this can be parallelized we anticipate that
the computation speed will largely be impacted by the number of dot products we
are performed rather than the complexity of each individual dot product. If we
use an equation based object such as a parametric sphere, we anticipate using
hard coded cross products of partial derivatives to get the surface normal
equation. Furthermore, we will neeed to calculate the lights irradiance which
will involve some form of integration either using sums which can be aided with
a technique such as prefix sum or using a more complex method such as Monte Carlo
integration which will still require some parallel reduction step.



\subsection*{Anticpated Obstacles}
% ANTICIPATED OBSTACLES
If we follow through with our plan to implement support for custom models or
materials, we may encounter issues depending on the material type. We first are
planning to mitigate this by not including support for such complexity until we
have a working version of our project on a simple, non-imported asset (such as
a sphere made with C code). Since there are several steps to the ray tracing
process, we will have to decide which steps are most important and which ones
we can afford to leave out.

\subsection*{Project Timeline}
% PROJECT TIMELINE
By the checkpoint, we expect to implement some form of backwards path traced
light ray detection for hitting objects. We also plan to start with implementing ray
tracing for solid (non-metallic) objects with standard normals. We may add
metallic objects later, but not first as they require solid objects to judge
how the ray tracing is working. We also plan to start with one fixed
point-light source. After the light ray collision detection, we will have to parallelize
the algorithm to compute the the intensity at the point. If time permits, we can add more
features to the rendering process such as material properties such as glossy/metallic
objects, microfacets (e.g. Cook-Torrance model for specular highlights), anti-aliasing,
and Ambient Occlusion for better shadow effects.


\end{document}