\documentclass[12pt]{article}

\usepackage{custom_package}
\usepackage[mocha,styleAll]{catppuccinpalette}
\usepackage[paper=letterpaper, vmargin=1in, hmargin=0.5in]{geometry}

% \fancyhead[L]{(Names)}
\fancyhead[C]{CSE 4059 (GPU Programming)}
% \fancyhead[R]{(IDs)}
\setlength{\headheight}{14.5pt}
\pagestyle{fancy}

\fboxsep=1mm
\fboxrule=2pt

\graphicspath{ {.} }

\begin{document}

{\section*{CUDA RAY TRACER (CRT)}}

{\subsection*{Project Abstract}}

% PROBLEM STATEMENT
Explore physically based rendering techniques involving ray tracing and accelerate the
rendering algorithms with CUDA, and potentially utilize OpenGL. We plan to first follow through a
CPU based ray tracing guide \href{https://raytracing.github.io/books/RayTracingInOneWeekend.html}{Raytracing In One Weekend}
and implement GPU-accelerated parallelization when possible, and utilize physics concepts to
approximate a realistic image render.

% OBJECTIVE
Our objective is to implement features that bring us closer to a photorealistic render of an object
i.e. a raytraced lighting/shading model, object material features, a camera/viewport to render the
image, and post-processing features. We plan to support dielectric and/or metallic materials.

%APPROACH
Implement Whitted style ray tracing algorithm to calculate the lighting.
We can potentially use tiled matrix multiply for implementing 3D projection like perspective.
Post-processing features like anti-aliasing and depth-of-field blur will probably use convolution.

% EXPECTED OUTCOME
We expect to speed up graphical rendering processes by utilizing the GPU when parallelization is
useful/necessary. Hopefully the final product will be able to to perform ray tracing and
realistically render simple objects and materials.

{\subsection*{Group Member List}}

\textbf{Names:}
\begin{itemize}
    \item Raj Sugavanam
    \item Junseo Shin
\end{itemize}

No differentiation in roles is expected. We will simultaneously work on
different parts of the implementation if needed, otherwise collaborate on
solving one.

\end{document}
