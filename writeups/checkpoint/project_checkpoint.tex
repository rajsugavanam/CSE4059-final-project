\documentclass[12pt]{article}


\usepackage{custom_package}
\usepackage[mocha,styleAll]{catppuccinpalette}
\usepackage[paper=letterpaper, vmargin=1in, hmargin=0.5in]{geometry}


% \fancyhead[L]{(Names)}
\fancyhead[C]{CSE 4059 (GPU Programming)}
\rhead{Raj Sugavanam \& Junseo Shin}
\lhead{CRT}
% \fancyhead[R]{(IDs)}
\setlength{\headheight}{14.5pt}
\pagestyle{fancy}


\fboxsep=1mm
\fboxrule=2pt


\graphicspath{ {.} }


\begin{document}


\begin{center}
    \section*{Project Checkpoint}
\end{center}


\subsection*{Progress Update}
%     \item \textbf{Current Progress:} [Describe what you have accomplished up to this point.]
%     \item \textbf{Comparison to Plan:} [Identify how your current progress matches or does not
%       match the progress you anticipated in your project plan.]

Currently we have successfully implemented a method to parse an \texttt{.obj} file and store the
mesh of triangles into the GPU memory. After the triangles are copied to the device memory, we are
able to correctly render the object using ray casting to determine the correct triangle 
intersection and interpolate the vertex normal to compute the normal map of each pixel being
rendered. Furthermore, we have implemented a basic ``lambertian'' shading model to
compute the color of the intersected triangle point based on the light source position.

This matches the timeline of our project plan which also gave us extra time to read up on
literature and useful libraries that will be paramount for the next steps of our project.

\subsection*{Implementation Details}
%   \item \textbf{Implementation:} [Describe the current state of your code implementation,
%       including any algorithms and optimizations you have applied.]
%   \item \textbf{Performance Measurements:} [Present appropriate performance measurements of your
%       code implementation to date.]
% [GitHub Repository Link]

\paragraph{Implementation:}

\begin{itemize}
    \item Our lighting model assumes that we have a fixed-direction light ray at every point in space. This is highly
        unrealistic as there is no `source' of light, however we have gotten a base lighting model to work. We implemented
        the M\"oller-Trombore triangle intersection algorithm, which is performed \(O(N)\) times, where \(N\) is proportional
        to the number of triangles (which is proportional to the number of threads).
\end{itemize}

\subsubsection*{Performance Measurements:}

Runtime Measurements:
\begin{itemize}
    \item 960 triangle mesh runtime: 13.685 ms
    \item 4753 triangle mesh runtime: 66.592 ms
    \item 18192 triangle mesh runtime: 248.574 ms
\end{itemize}

NCU Measurements:
\begin{itemize}
    \item SOL Compute (SM) Throughput: 96.41\%
    \item Achieved Occupancy: 96.20\%
\end{itemize}

\href{https://github.com/rajsugavanam/CSE4059-final-project}{https://github.com/rajsugavanam/CSE4059-final-project}

\subsection*{Project Plan Revisions}
%   \item \textbf{Revisions:} [Detail any changes to your project plan.]
%   \item \textbf{Justification:} [Explain why these revisions are necessary.
%       It is common and expected for projects to change direction slightly during implementation.]

\paragraph{Revisions:}

Clearer Objectives have been set for the next steps of our project:

\begin{itemize}
    \item Cornell Box Scene
    \item Monte Carlo Path tracing and Lambertian Diffuse Reflection
\end{itemize}

We also plan to put a ground surface for the floor, and plan on making a triangle-based light source. We may use
multiple objects to test multiple bounces of a ray, compared to our original decision in the project plan to use only
one model. To this effect, we will need to make a new \texttt{class} for light sources. We also change the way our
memory layout of our vertex or normal list works to allow for better coalescing.

\paragraph{Justification:}

\begin{itemize}
    \item The Cornell Box Scene will allow use to test the path tracing algorithm as it requires
        multiple bounces of a ray to be able to compute the color of a pixel after it hits a light
        source.
    \item This will require us to to use the cuRAND to uniformly sample the reflected
        ray at a bounce point which must point in the hemisphere of the triangle normal.
        The lambertian diffuse reflection can be optimized by a cosine weighted sampling 
        so we don't have to reject a uniform ray sample that does not fit in the hemisphere.
    \item Using triangle meshes to represent both the light source and ground/walls of the
        scene will allow us to use the same intersection algorithm which will simplify the
        control flow of the program.
    \item Changing the memory layout to using structure of arrays (SoA) is necessary for
        parallel computing which should in theory speed up our program.
\end{itemize}

\end{document}
