\documentclass[../main.tex]{subfiles}

\graphicspath{{../images/}}
\ifSubfilesClassLoaded{
    \twocolumn
}{}

\begin{document}

\section{Approach}

\emph{Model setup.} We begin by setting up a simple perspective camera and read a sequence of object files. 
We include a scene manager to handle multiple objects. We consider distances of
triangles to the camera during rendering to ensure the output looks correct.

\emph{Path Tracing Strategy.} We follow a widely used method in this respect by
tracing light backwards, ie from the camera to the light source. As fully
accurate ray tracers are almost impossible to feasibly run, our approach made
heavy use of estimation techniques.

\emph{Color.} During our implementation we found an opportunity to use
wavelength-based colors rather than only RGB. This implies the usage of
conversions from wavelength to RGB, as monitors can only feasibly display RGB.
To this effect, we experimented with different color representations.
% \cite{IEC61966-2-1-Amd1}
% \url{https://www.iec.ch/}

% for subfile compilation
\ifSubfilesClassLoaded{%
    \nocite{*}
    \bibliographystyle{ACM-Reference-Format}%
    \bibliography{references}%
    \twocolumn
}{}
\end{document}
