\documentclass[../main.tex]{subfiles}

\graphicspath{{../images/}}
\ifSubfilesClassLoaded{
    \twocolumn
}{}

\begin{document}

\section{Results}

We direct the reader to Figure (\ref{fig:teaser}) for the output \texttt{.ppm}
images for our program. Figures (\ref{fig:teaser1}) and (\ref{fig:teaser2})
indicate our representations for AABBs and smooth shading. Smooth shading is a
technique that calculates interpolated normals at any point on a triangle,
using vertex normals. That is, if we represent point \(P\) in space with barycentric coordinates of
triangle \(ABC\) by
\begin{equation*}
    P = \lambda_1 A + \lambda_2 B + \lambda_3 C,
\end{equation*}
then we calculate the surface normal \emph{at point} \((\lambda_1, \lambda_2, \lambda_3)\) to be
\begin{equation*}
    N = \lambda_1 N_i + \lambda_2 N_j + \lambda_3 N_k,
\end{equation*}
where \(N_i\), \(N_j\), and \(N_k\) are vertex normals of \(A\), \(B\), and \(C\) respectively.

If we had used a face normal based method, where each triangle gets assigned \emph{one} \(N\)
regardless of any barycentric point \((\lambda_1, \lambda_2, \lambda_3)\), we would see
`sharp' shading with more pronounced edges.

As RGB color can be encoded as a 3-dimensional vector, we see that the surface color of Figure
(\ref{fig:teaser2}) represents the direction of our calculated \(N\).

Soft shadows, seen in Figures (\ref{fig:teaser3}) and (\ref{fig:teaser4}), are
a result of our ray tracing model. Figure (\ref{fig:teaser3}) has a far greater
amount of noise due to an extremely low sample count, which cannot rely on the
law of large numbers to accurately color an output pixel with volume of light
paths.
% \cite{IEC61966-2-1-Amd1}
% \url{https://www.iec.ch/}

% for subfile compilation
\ifSubfilesClassLoaded{%
    \nocite{*}
    \bibliographystyle{ACM-Reference-Format}%
    \bibliography{references}%
    \twocolumn
}{}
\end{document}
