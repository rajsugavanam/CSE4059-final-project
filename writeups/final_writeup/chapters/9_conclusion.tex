\documentclass[../main.tex]{subfiles}

\graphicspath{{../images/}}
\ifSubfilesClassLoaded{
    \twocolumn
}{}

\begin{document}

\section{Conclusion}

One approach we considered was to use the \texttt{curand} interface to generate
random numbers. We had a partial implementation of this, but ended up using our
own XORShift pseudo-random number generator. We still sample with a
cosine-weighted distribution. We could have seen further optimization from
usage of RT cores via NVIDIA Optix (speed-up with ray-triangle intersection,
etc). We discussed implementing more material properties for objects;
specifically, reflective, specular, and caustic. With enough time, we may have
developed a protocol to utilize corresponding \texttt{.mtl} files, which we
currently ignore.

Overall,
our model is successful in feasibly simulating the basic behavior of light.
Further developments would focus on enhancing the visual complexity of
our renders.
% \cite{IEC61966-2-1-Amd1}
% \url{https://www.iec.ch/}

% for subfile compilation
\ifSubfilesClassLoaded{%
    \nocite{*}
    \bibliographystyle{ACM-Reference-Format}%
    \bibliography{references}%
    \twocolumn
}{}
\end{document}
