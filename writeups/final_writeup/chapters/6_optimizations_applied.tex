\documentclass[../main.tex]{subfiles}

\graphicspath{{../images/}}
\ifSubfilesClassLoaded{
    \twocolumn
}{}

\begin{document}

\section{Optimizations}

One primary optimization we used was AABBs (axis-aligned bounding boxes) to
compute a top-level intersection box that rays which have a chance to strike a
triangle are guaranteed to hit. The intersection of this box is low cost and
far easier to initially compute than iterate through every mesh triangle,
therefore we use it as an initial filter to determine which pixels should
proceed to multi-bounce path trace. This improves our compute time for
smaller/farther distance meshes. The effect of this optimization reduces as
more of the mesh fills the viewing range of the camera.

We restructured the arrays of vectors for a mesh to prioritize components first, rather
than vectors first. Our array structure will use a major order with all vertices' \(x\)-components
first, then \(y\), then \(z\). This makes coalescing memory accesses between threads far easier.
% \cite{IEC61966-2-1-Amd1}
% \url{https://www.iec.ch/}

% for subfile compilation
\ifSubfilesClassLoaded{%
    \nocite{*}
    \bibliographystyle{ACM-Reference-Format}%
    \bibliography{references}%
    \twocolumn
}{}
\end{document}
