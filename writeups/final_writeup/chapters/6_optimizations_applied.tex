\documentclass[../main.tex]{subfiles}

\graphicspath{{../images/}}
\ifSubfilesClassLoaded{
    \twocolumn
}{}

\begin{document}

\section{Optimizations}

One primary optimization we used was AABBs (axis-aligned bounding boxes) to
compute a top-level intersection box that rays which have a chance to strike a
triangle are guaranteed to hit. The intersection of this box is low cost and
far easier to initially compute than iterate through every mesh triangle,
therefore we use it as an initial filter to determine which pixels should
proceed to multi-bounce path trace. This improves our compute time for
smaller/farther distance meshes. The effect of this optimization reduces as
more of the mesh fills the viewing range of the camera.

% CUDA stuff

We also used several optimizations exclusive to the CUDA programming model. Due to the
complex nature of simulating physical light, several parts of our code were written with
object-oriented programming in mind. This requires us to enable separable compilation and
linking device code to the host code. One caveat of this is that having separately compiled
device code requires device link time optimization to be enabled via the
\texttt{CMAKE\_INTERPROCEDURAL\_\\OPTIMIZATION} option in CMake. This allows the compiler to
optimize the device code at link time rather than at compile time, thus allowing 
function inlining and other optimizations to be performed across translation units.

For the AABB calculation, we first used a Structure of Arrays (SoA) memory layout to allow for
memory coalescing we perform reduction on shared memory to calculate the minimum and maximum bounds
of a triangle mesh. Furthermore, due to the dynamic nature of the AABB calculation, our reduction
implementation allows for an arbitrary length array of triangles to be passed in, and completes
the final reduction between the blocks of a single kernel using custom \texttt{atomicMin()} and
\texttt{atomicMax()} implemented based on the CUDA \texttt{atomicCAS()} function. Finally,
CUDA streams were used to allow for multiple reduction to be performed in parallel on the GPU
since the AABB calculation requires 9 reductions (each triangle has three vertices each with an
$x, y, z$ component stored as a \texttt{float*}) to be performed on a triangle mesh object. Although
the SoA layout greatly improves AABB calculation performance, the memory accesses during the
traversal of a rays through a scene is not coalesced very well since the path of each ray
diverges greatly, so moving from using a Array of Structures (AoS) to a SoA layout only improved
the ray tracing performance by about 1.12x. 

Since ray tracing is a highly memory bound operation, we used constant memory to store
the material properties of each triangle mesh object. This included the reflectance spectrum
of each triangle mesh object to be accessable via \texttt{reflectance\_id} and the emission spectrum
of the light object. In addition, we also used constant memory to store the discrete values of
the color matching functions (CMFs) to compute the CIE 1931 XYZ color space values and its
corresponding XYZ to RGB transformation matrix. This greatly reduces the number of
global memory accesses to the GPU already bounded by the triangle intersection test.

% for subfile compilation
\ifSubfilesClassLoaded{%
    \nocite{*}
    \bibliographystyle{ACM-Reference-Format}%
    \bibliography{references}%
    \twocolumn
}{}
\end{document}
